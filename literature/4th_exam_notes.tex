% Created 2022-04-30 Sat 17:49
% Intended LaTeX compiler: pdflatex
\documentclass[11pt]{article}
\usepackage[utf8]{inputenc}
\usepackage[T1]{fontenc}
\usepackage{graphicx}
\usepackage{longtable}
\usepackage{wrapfig}
\usepackage{rotating}
\usepackage[normalem]{ulem}
\usepackage{amsmath}
\usepackage{amssymb}
\usepackage{capt-of}
\usepackage{hyperref}
\author{Julian Eng}
\date{\today}
\title{}
\hypersetup{
 pdfauthor={Julian Eng},
 pdftitle={},
 pdfkeywords={},
 pdfsubject={},
 pdfcreator={Emacs 28.1 (Org mode 9.5.2)}, 
 pdflang={English}}
\begin{document}

\tableofcontents

\section{Important things to remember}
\label{sec:orgaa226f5}
\subsection{What is a key?}
\label{sec:org31faaa5}
A key is the \emph{minimum} set of attributes that \emph{all} other attributes depend
on.

IF WE REMOVE AN ATTRIBUTE FROM THE KEY, THEN WE LOSE SOME ATTRIBUTE IN THE
DEPENDENCY.

Ex: we have \(A \to B, C, D; \quad B \to E;\quad F \to G;\quad A, F \to H\)
\begin{itemize}
\item \(A, F\) is the key. If we removed \(F\), then we would lose \(G\) being
dependent on \(F\), and if we removed \(A\), we'd lose everything else.
\end{itemize}

\subsection{Superkeys}
\label{sec:org69e2255}
\begin{itemize}
\item If you apply the inference rules on them, we should get all the attributes
of the relation, because there shouldn't be any attributes that don't
eventually depend on a superkey.
\end{itemize}

\section{Decomposition}
\label{sec:org93773c0}
\begin{itemize}
\item Important things we want:
\begin{itemize}
\item Attribute Preservation
\item Dependency Preservation
\item Lossless (nonadditive) Join Property
\end{itemize}
\end{itemize}

\subsection{Functional Dependency}
\label{sec:org541d419}
\begin{itemize}
\item \(Y\) is functionally dependent on \(X\) in \(R\) if for every \(x \in R.X\), there
is exactly one \(y \in R.Y\).
\end{itemize}

\subsection{Full Functional Dependency}
\label{sec:org0fd9761}
\begin{itemize}
\item \(Y\) is FULLY functionally dependent on \(X\) in \(R\) if it is functionally
dependent on \(X\) and NOT functionally dependent on any proper subset of
\(X\).
\end{itemize}

\subsection{Transitive Dependency}
\label{sec:orge979657}
\begin{itemize}
\item \(Y\) is transitively dependent on \(X\) in \(R\) if there exists set of
attributes \(Z\) that is a) NOT A CANDIDATE KEY and b) NOT A SUBSET OF
ANY KEY OF \(R\), such that \(X \to Z\) and \(Z \to Y\) hold.
\end{itemize}


\subsection{Normal Form Rules:}
\label{sec:orgcc52147}
\subsubsection{1NF}
\label{sec:orgbb4fcfd}
All values are atomic (no sets, tuples, or relations in relations)
\subsubsection{2NF}
\label{sec:org5a4b8d4}
Relation is in 1NF AND every nonkey attribute is fully functionally
dependent on the key
\subsubsection{3NF}
\label{sec:orgde76511}
Relation is in 2NF and every nonkey attribute is nontransitively dependent
on the key
\subsubsection{BCNF}
\label{sec:orga46adb6}
Every determinant is a candidate key

Practice exam vs. lecture difference:

Lecture has \(Email \to CurrentCity, Salary;\; CurrentCity \to Salary\) as
2NF.
This makes sense because \(Salary\), while dependent on \(Email\) by itself,
also depends on it through \(CurrentCity\), which is a nonkey.


Practice exam has the equivalent of \(Email \to CurrentCity, Salary;\; Email,
    CurrentCity \to Salary\) as BCNF.
This makes sense because \(Salary\) now depends on both \(Email, CurrentCity\)
at once, but \(Email, CurrentCity\) is a candidate key, given that it
determines \(Salary\), so this doesn't count as a transitive dependency.




\subsection{Closure of Attributes w.r.t A Dependency Set}
\label{sec:orgea6c8ea}
\begin{itemize}
\item This is all the attributes that depend on the given attributes, based on
the dependency set.
\end{itemize}

\section{Important Memory Things}
\label{sec:orgdd05e16}
\begin{itemize}
\item Blocking factor = number of records per block = \texttt{block size / record size}
\item Copy time = \texttt{seek time + rotation delay + block copy time * numblocks}
\item Files are bunches of blocks linked in a linked-list of pointers
\item Unspanned = records don't get split across blocks
\begin{itemize}
\item When there's a choice, we go with unspanned
\item If the object is too big for a block, then obviously we must span it
\end{itemize}
\item Bulk transfer: when finding a block, find blocks after it too to save on
seek costs
\end{itemize}

\subsection{Heap}
\label{sec:orga93fc13}
\begin{itemize}
\item No organization, records are just put into blocks in the order they are
inserted.
\item To find something, takes on average \(N/2\) time
\end{itemize}

\subsection{Sorted}
\label{sec:org93a8244}
\begin{itemize}
\item Records are sorted by key
\item Binary search: \(\log_2{N}\) cost
\end{itemize}

\subsection{Sorted w/ index}
\label{sec:org94f446c}
\begin{itemize}
\item Records are sorted by key
\item We build a (sparse) index; entries are keys of the FIRST record in a block
\item Binary search: \(\log_2{N} + 1\) cost, where \(N\) is the number of index
blocks.
\begin{itemize}
\item \(+ 1\) is for accessing the data block itself.
\end{itemize}

\item A DENSE index has all the keys for data, which will still live in fewer blocks.

\item When considering index size and fanout, you must keep in mind the data type
of the index, and the size of a block pointer. For example, if the data
type is a \texttt{varchar28} and the block pointer is 4B, then each index entry
costs 32B. If we're filling a block of size 4000B 80\% of the way, we have
3200B to fill with index entries. 3200 / 32 = 100, so our fanout is 100.

\begin{itemize}
\item Fanout refers to the number of index entries per index block.
\end{itemize}

\item \texttt{number of data blocks / fanout = number of index blocks}

\item Primary indices are good for point queries (single access) and range queries
(facts about many records in sequence)

\item Clustered index defines the order the table is laid out in
\end{itemize}

\subsection{Secondary Index}
\label{sec:org8facf41}
\begin{itemize}
\item Values are not sorted on this field, so we need to grab all the values from
the records, THEN sort them.
\item If this isn't a key field, then we have to choose whether or not to keep
multiple pointers to the block in question.
\end{itemize}

\subsection{Multilevel Index}
\label{sec:org0c8e3a7}
\begin{itemize}
\item I herd you liek indexes
\item Lookup is now \(\log_{fanout}{n} + 1\) where \(n\) is the number of index blocks
\end{itemize}

\subsection{Hashing}
\label{sec:orgeea01fe}
\begin{itemize}
\item Each bucket can have a bunch of blocks, generally we'll assume that the
hash function uniformly distributes blocks. After that it's just unit
conversion.
\end{itemize}
\end{document}