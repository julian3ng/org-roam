% Created 2022-04-02 Sat 15:12
% Intended LaTeX compiler: pdflatex
\documentclass[11pt]{article}
\usepackage[utf8]{inputenc}
\usepackage[T1]{fontenc}
\usepackage{graphicx}
\usepackage{grffile}
\usepackage{longtable}
\usepackage{wrapfig}
\usepackage{rotating}
\usepackage[normalem]{ulem}
\usepackage{amsmath}
\usepackage{textcomp}
\usepackage{amssymb}
\usepackage{capt-of}
\usepackage{hyperref}
\author{Julian Eng}
\date{\today}
\title{SQL}
\hypersetup{
 pdfauthor={Julian Eng},
 pdftitle={SQL},
 pdfkeywords={},
 pdfsubject={},
 pdfcreator={Emacs 27.2 (Org mode 9.4.4)}, 
 pdflang={English}}
\begin{document}

\maketitle
\tableofcontents

\section{SQL}
\label{sec:org5db7974}
\subsection{SELECT}
\label{sec:org0035aa1}
\texttt{SELECT <fields> FROM <table> WHERE <conditions>}
\subsection{INSERT}
\label{sec:org52bf0ea}
\texttt{INSERT INTO <table> (field1, field2, field3) VALUES (v1, v2, v3)}
\subsection{UPDATE}
\label{sec:org94ab9c5}
\texttt{UPDATE <table> SET f1=v1,f2=v2,f3=v3 WHERE <condition>}
\subsection{Generally\ldots{}}
\label{sec:org4721477}
\begin{verbatim}
SELECT column1, column2, ..., columnn
FROM table1, table2, ..., tablem
WHERE condition;
\end{verbatim}

Joining tables by commas will give you a cartesian product.

The general form projects the given columns on the rows indicated by the
condition from the cartesian product of the given tables.
\subsection{DISTINCT}
\label{sec:orge313d34}
\texttt{SELECT DISTINCT <columns> FROM <table> WHERE <condition>}

The \texttt{DISTINCT} keyword will deduplicate the result of the query run without
the \texttt{DISTINCT}.
\subsection{NATURAL JOIN}
\label{sec:org95703e0}
\begin{verbatim}
SELECT a.field1, b.field1, a.field2
FROM A a, B b
WHERE a.field2 = b.field2;
\end{verbatim}

The \texttt{FROM} clause creates a catesian product and the \texttt{WHERE} clause creates
the join condition. 

\begin{verbatim}
SELECT a.field1, b.field1, a.field2
FROM A AS a NATURAL JOIN B AS b;
\end{verbatim}

Same as above. If there are no matching join fields, this is the same as
cartesian product.
\subsection{LEFT OUTER JOIN}
\label{sec:org7e06efb}
\begin{verbatim}
SELECT A.field1, B.field1, B.field2
FROM A LEFT OUTER JOIN B;
\end{verbatim}

This will join A to B on their matching column; if the column in A doesn't
have a match in B, B's columns in the output will be NULL.
\subsection{String Matching}
\label{sec:org71dc14f}
\texttt{SELECT <columns> FROM <table> WHERE field LIKE 'Foo\%';}

\begin{itemize}
\item \texttt{\%} matches any string (regex \texttt{.*})
\item \texttt{\_} matches any one character (regex \texttt{.})
\end{itemize}
\subsection{Sorting}
\label{sec:org9536da3}
\texttt{SELECT <columns> FROM <table> ORDER BY field ASC;}

(can replace \texttt{ASC} with \texttt{DESC} for reverse order)
\subsection{UNION}
\label{sec:org991e710}
\begin{verbatim}
<SELECT QUERY>  UNION <SELECT QUERY>
\end{verbatim}

\begin{itemize}
\item All rows of the first query set union with all rows of the second query
\item DUPLICATES ARE NOT PRESERVED
\item If you want to preserve duplicates, use \texttt{UNION ALL}
\end{itemize}
\subsection{INTERSECT}
\label{sec:org400d824}
\begin{itemize}
\item Same as \texttt{UNION} but it's an intersection.
\item \texttt{INTERSECT ALL} behaves the same way. (What happens if there's two in the
first query but only one in the second query?)
\end{itemize}
\subsection{EXCEPT}
\label{sec:org2c37d82}
\begin{itemize}
\item Same as \texttt{UNION} but it's a set difference.
\item \texttt{EXCEPT ALL} makes the output contain the difference in occurances of the
rows in question.
\end{itemize}
\subsection{Builtin functions}
\label{sec:orgf0ac895}
\begin{itemize}
\item \texttt{COUNT}, \texttt{SUM}, \texttt{AVG}, \texttt{MIN}, \texttt{MAX}
\item \texttt{SELECT COUNT(*) FROM Table;} -> create one row with the number of rows in
the table.
\end{itemize}
\subsection{GROUP BY}
\label{sec:orgbe4a5db}
\begin{verbatim}
SELECT col, COUNT(*) as num, AVG(numcol) av
FROM Table
GROUP BY col
ORDER BY num ASC;
\end{verbatim}

\begin{itemize}
\item \texttt{col} must appear in the output because it's being grouped on.
\item the \texttt{COUNT} and \texttt{AVG} will act on the groups, not the entire table.
\end{itemize}
\subsection{HAVING}
\label{sec:org23ac478}
\begin{verbatim}
SELECT col, COUNT(*) as num, AVG(numcol) av
FROM Table
GROUP BY col
HAVING num > 3
ORDER BY num ASC;
\end{verbatim}

\begin{itemize}
\item \texttt{HAVING} will create a condition on the returned groups. In this case, we
restrict outputs to groups with more than 3 members.
\end{itemize}
\subsection{Nesting Queries}
\label{sec:org573b37a}
\subsubsection{IN/NOT INT}
\label{sec:orgdbc0b41}
\begin{verbatim}
SELECT Foo, Bar
FROM Table
WHERE Foo IN
      (SELECT Foo
       FROM Table2
       WHERE Baz = 'qux');
\end{verbatim}

\begin{itemize}
\item Inner query makes a table that the outer query searches on.
\item The inner query has no relation to the outer query besides providing a
table to search.
\end{itemize}
\subsubsection{=, \(\neq\), \(\ge\), \(\le\), < , > , SOME/ALL}
\label{sec:orgfc3b78a}
\begin{verbatim}
SELECT Foo
FROM R, Y
WHERE R.B = Y.B AND Z > ALL
      (SELECT Z
       FROM R, Y
       WHERE R.B = Y.B AND R.A = 'Bar');
\end{verbatim}

\begin{itemize}
\item Again, inner query makes a table that the outer query does something with.
\item In this case, it compares each row's \texttt{Z} to \textbf{every} \texttt{Z} in the inner
table.
\end{itemize}
\subsubsection{Correlated Queries}
\label{sec:org6a7a542}
\begin{verbatim}
SELECT R.E, R.Y
FROM R
WHERE NOT EXIST
      (SELECT *
       FROM U
       WHERE U.E = R.E);
\end{verbatim}

\begin{itemize}
\item In this case, the inner query refers to something from the outer query
(\texttt{R.E})
\item Think of it as the inner query getting evaluated for each row of the outer query
\end{itemize}
\end{document}